\recipe[style=style2]{Rumaki}

\info[servings=24,
		time = 40, 
		energy = 40 (each),
		cost =  25,
		urlsource = http://allrecipes.com/recipe/235095/easy-rumaki-with-pineapple/]{}

\begin{ingredients}
	\ingredient{24}{}{pinapple (cubes)}
	\ingredient{24}{}{chestnut (slices)}
	\ingredient{8}{}{bacon slices (thick)}
	\ingredient{120}{ml}{sesame-ginger salad dressing}
	\ingredient{150}{ml}{white wine (dry)}
	\ingredient{100}{g}{green onions}
\end{ingredients}

\begin{preparation}
	\step Preheat oven to 375 degrees F (190 degrees C). Line the bottom section of a broiler pan with aluminum foil, top with the broiler rack, and spray rack with cooking spray.
	
	\step Place a water chestnut slice atop each pineapple cube; wrap each with 1 bacon slice, securing with a toothpick. Arrange wrapped pineapple on the prepared broiler rack. 
	
	\step Bake in the preheated oven for 7 minutes; turn and continue baking until bacon is almost crisp, about 8 more minutes. Brush rumaki with sesame-ginger dressing and continue baking until bacon is crisp, about 5 more minutes. Garnish rumaki with green onion. 
\end{preparation}

\begin{notes}
	\note{This recipe uses \texttt{style2} header and the \texttt{ingredient} environment. It also shows the cost.}
	\note{There is also a picture, using \texttt{recipefigure} with the \texttt{wide} style.}
\end{notes}

\recipefigure[style = wide, height = 0.5]{rumaki.jpg}